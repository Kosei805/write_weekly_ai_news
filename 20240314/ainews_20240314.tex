\documentclass[dvipdfmx,12pt]{beamer}% dvipdfmxしたい
\usepackage{bxdpx-beamer}% dvipdfmxなので必要
\usepackage{pxjahyper}% 日本語で'しおり'したい
\usepackage{minijs}% min10ヤダ
\usepackage{amsmath,amssymb,amsthm}% 数式とか
\usepackage{graphicx}% 画像とか
\usepackage{tikz}% 図とか
\usepackage{ascmac}% 枠とか
\usepackage{url}% URLとか
\usepackage{here}% 画像とか
\usepackage{hyperref}% ハイパーリンクとか
\renewcommand{\kanjifamilydefault}{\gtdefault}% 既定をゴシック体に

% あとは欧文の場合と同じ
\usetheme{Copenhagen}% https://hartwork.org/beamer-theme-matrix/
\usecolortheme{default}% https://hartwork.org/beamer-theme-matrix/

\AtBeginSection[]{
    \frame{\tableofcontents[currentsection, hideallsubsections]} %目次スライド
}

\title{今週のAINews}
\author{中園康聖}
\date{2024/03/14}
\setbeamertemplate{caption}[numbered] % 図表番号をつける
\begin{document}
\bibliographystyle{junsrt}% 参考文献のスタイル(https://mathlandscape.com/latex-bibstyles/)
\begin{frame}
\titlepage
\end{frame}
\begin{frame}{目次}
\tableofcontents
\end{frame}
\section{Claude3}
% \subsection{サブセクション名}
\begin{frame}
\frametitle{claude3の登場}
新たにClaude3が登場した。登場したモデルは以下の3つである。

\begin{itemize}
\item Opus
\item Sonnet
\item Haiku
\end{itemize}
\end{frame}

\begin{frame}
\frametitle{Haiku}
GPT3.5より安価なAPIモデル。もし、クオリティよりも低コストであることを求めるならこれがベスト。
\end{frame}

\begin{frame}
\frametitle{Sonnet}

ブラウザ版で利用できるモデル。話題にはあまり上がっていない。

\end{frame}


\begin{frame}
\frametitle{Opus}
最も高性能なモデルで、現在最も性能の良いGPT4を上回った。
IQが100を超えると言われている。コンテキスト長が長く、ブラウザ版でpdfを渡せるため、データを参照しやすい。
英語性能はGPT4が強く、日本語性能はClaude3が強い。今度、日本で\href{https://jawsug-ai.connpass.com/event/313318/}{イベント}をするらしい。
\begin{itemize}
\item \href{https://twitter.com/sou_btc/status/1767825425269666093?s=12&t=HWsH9aiIPwtM8W1pjX9WBA}{2chスレッドの作成}
\item \href{https://twitter.com/shi3z/status/1766765085488472306?s=12&t=HWsH9aiIPwtM8W1pjX9WBA}{アニメーションの作成}
\item \href{https://note.com/genkaijokyo/n/n186c061b7476}{当直表を作成}
\item \href{https://zenn.dev/olemi/articles/a8b492712fd9e7}{動画解析}
\end{itemize}
\end{frame}

\begin{frame}
\frametitle{最後に}
ここまで、Claude3に付いて説明してきたが、そろそろGPT5が登場してまた逆転するかもしれない。
MicrosoftのページにGPT5という記載が現れた。
また、今日(3/14)はGPT4が登場した日。ちょうど一年目である。
なんなら、今日新しい発表があるかも…
\end{frame}


% \begin{frame}[allowframebreaks]
% \bibliography{sample} %参考文献ファイル名(sample.bib) 拡張子は指定しない
% \end{frame}
\end{document}